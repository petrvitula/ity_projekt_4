\documentclass[a4paper, 11pt]{article}

\usepackage[czech]{babel}
\usepackage[utf8]{inputenc}
\usepackage[left=2cm, top=3cm, text={17cm, 24cm}]{geometry}
\usepackage{times}
\usepackage[unicode]{hyperref}
\usepackage{amsmath}
\hypersetup{colorlinks = true, hypertexnames = false}

\begin{document}

\begin{titlepage}
        \begin{center}
            \Huge
			\textsc{Vysoké učení technické v~Brně} \\
			\huge
			\textsc{Fakulta informačních technologií} \\
			\vspace{\stretch{0.382}} % zarovnani na fibo
			\LARGE
			Typografie a~publikování\,--\,4.~projekt \\
			\Huge
			Bibliografické citace
			\vspace{\stretch{0.618}} % zarovnani na fibo
		\end{center}

		{\Large
			\today
			\hfill
			Petr Vitula
		}
\end{titlepage}

\section{Systém \LaTeX}

\subsection{Co je to \LaTeX ?}

\LaTeX{} je především prostředí/systém pro tvorbu dokumentů, můžete to chápat jako obdobu pro souborový formát jako je Word, LibreOffice či HTML. Je to i jeden z nejstarších způsobů (formátů) pro tvorbu dokumentů (je založeny na systému pro formátování dokumentů \TeX z roku 1978). Zajímavý je tím, že k vytváření složitě formátovaných dokumentů vám bude stačit běžný textový editor.\cite{365tipu2018}


\subsection{Proč používat \LaTeX ?}

Obrovská síla \LaTeX u se projevuje zejména při psaní matematických vzorců a při vytváření matematických dokumentů obecně. Psaní vzorců v programech s grafickým rozhraním (Word, OO Writer, atd.) je velmi zdlouhavé kvůli nutnosti používat myš pro výběr z rozsáhlých menu. V LaTeXovém dokumentu jsou naproti tomu formátovací značky přímo součástí textu a ač se to na první pohled nezdá, je mnohem pohodlnější pamatovat si několik textových značek, než pozice elementů v grafickém menu.

TeX i LaTeX vynikají obrovskou přesností. Když v dokumentu specifikujeme, že mezi dvěma slovy má být mezera 1 centimetr, můžeme si být jisti, že po vytištění bude na papíře mezera přesně takto široká.

Další obrovskou výhodou (která často předčí i možnost sázet matematické vzorce) je možnost vytváření vlastních maker. Výroba běžných maker není vůbec složitá a v případě potřeby máme k dispozici velmi silný vyjadřovací prostředek.\cite{Martinek2010}

\subsection{Instalace}

Pro práci tedy potřebujeme dvě věci: program na překlad souborů .tex (ze vstupního souboru vytvoří výstupní) a textový editor na psaní souborů .tex. Pokud máte počítač s Windows postupujte takto:

\begin{itemize}
    \item Na překlad si nainstalujte TexLive zde (klikněte na odkaz install-tl-windows.exe).
    \item Jako editor si nainstalujte zase Texmaker zde.
\end{itemize}

Více informací jak postupovat při instalaci najdeme zde \cite{Vyfuk2024}
\noindent Texlive zabírá cca 3 GB, neboť obsahuje všechna možná rozšíření zvaná balíčky, které mnohdy obsahují svá vlastní písma nebo glyfy. Po instalaci tento program zůstane na vašem disku a v pozadí bude dělat svou práci, už se o něj nikdy nebudete muset starat.

\noindent Texmaker je program, ve kterém se budete pohybovat při sázení dokumentů a budete s ním nejvíce pracovat.

\subsection{Styly stránek}
Systém \LaTeX \hspace{0.1em} nabízí tři předdefinované kombinace záhlaví/paty stránek --- tzv. stránkové styly. Parametr style příkazu: 
\vspace{12pt}

\verb|\pagestyle{style}|
\vspace{12pt}

\begin{flushleft}
definuje, který ze stránkových stylů se užije. Předdefinované styly stránek mohou být, jak je uvedeno v \cite{Zmitko2015}:
\end{flushleft}


\begin{itemize}
    \item plain - tiskne čísla stránek na spodním okraji stránky ve středu paty stránky. Toto je základní stánkový styl.
    \item headings - tiskne jméno aktuální kapitoly a číslo stránky v záhlaví každé stránky a pata stránky zůstává prázdná.
    \item empty - nastavuje prázdné záhlaví i patu stránky.
\end{itemize}


\subsection{Práce v \LaTeX u}

Práce se systémem \LaTeX{} připomíná programování, jelikož zahrnuje tři hlavní kroky, jak je popsáno v \cite{Rybicka2003}:
\begin{enumerate}
\item psaní (úprava) zdrojového textu,
\item překlad -- vysázení,
\item prohlížení.
\end{enumerate}

\subsection{Matematické prostředí }

Základní prvky a výrazy zapisujeme většinou intuitivní stejným způsobem,
jako bychom je psali například rukou či v jiném plain-text editoru. To znamená, že například pro mocniny užíváme zápis ve tvaru $ a ^ b$, odmocniny
zapisujeme $\sqrt[x]{y}$, zlomky $\frac{x}{y}$ a takto bychom mohli pokračovat dále. Většina těchto výrazů je zcela intuitivní a jistý nebude dělat
nikomu žádný problém. Další typy jak dokážeme zobrazit matematické záležitosti najdeme zde. \cite{Vesely2008} Můžeme jednoduše sázet všechny typy závorek, znaky a symboly viz. \cite{Olsak2014}


\subsection{Algoritmy}
Pomocí \LaTeX u máme možnost prezentovat algoritmy v pochopitelném prostředí i pro začátečníky viz. \cite{Erickson2019}
\vspace{1em}

$ \begin{aligned}
    &\text { \underline{WhoTargetsWhom} }(H t[1 \ldots n]) \text { : }\\
    &\text { for } j \leftarrow 1 \text { to } n\\
    &\langle\langle\text { Find the left target } L[j] \text { for hero } j\rangle\\
    &L[j] \leftarrow \text { NONE }\\
    &\text { for } i \leftarrow 1 \text { to } j-1\\
    &\text { if } H t[i]>H t[j]\\
    &L[j] \leftarrow i\\
    &\text { (Find the right target } R[j] \text { for hero } j\rangle\rangle\\
    &R[j] \leftarrow \text { NoNE }\\
    &\text { for } k \leftarrow n \text { down to } j+1\\
    &\text { if } H t[k]>H t[j]\\
    &R[j] \leftarrow k\\
    &\text { return } L[1 . . n], R[1 . . n]
\end{aligned} $


\subsection{Logika}
Podobně jako algoritmy můžeme zobrazit i logické příklady a různé další operace viz \cite{Bagheri2024}\vspace{1em} 

If $\Gamma(\bar{x})$ is a face of $S_{\bar{x}}(T)$ and $\bar{y}$ is disjoint from $\bar{x}$, then $\Gamma(\bar{x}, \bar{y}) = \Gamma(\bar{x})$ is a face of $S_{\bar{x} \bar{y}}(T)$.

If $\Gamma_i$ is a face of $S_{\bar{x}}(T)$ for each $i \in I$, then so is $\bigcup_{i \in I} \Gamma_i$ (if satisfiable).

If $\Gamma$ is a face and $\Gamma \models \theta(\bar{x}) \leq 0$, then $\Gamma \cup \{0 \leq \theta(\bar{x})\}$.


\subsection{Rovnice}
Fyzika \LaTeX u též nebude dělat problém, jako příklad můžeme uvést zobrazení rovnic. Další rovnice najdeme ve článku. \cite{YiXu2024}
\vspace{1em}

$
\vec{A}_{\kappa, k_z, m_\gamma, \lambda}^{v o}(\vec{r})=\int \vec{A}_{\vec{k}, \lambda}^{p l}(\vec{r}) a_{m_\gamma \kappa}\left(\vec{k}_{\perp}\right) \frac{d^2 \vec{k}_{\perp}}{(2 \pi)^2}
$

\newpage
	\bibliographystyle{czechiso}
	\renewcommand{\refname}{Literatura}
	\bibliography{proj4}

 
\end{document}
